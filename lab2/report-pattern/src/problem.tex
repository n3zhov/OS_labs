\section{Постановка задачи}

\CWProblem{
Составить и отладить программу на языке Си, осуществляющую работу с процессами и взаимодействие между ними в одной из двух операционных систем. В результате работы программа (основной процесс) должен создавать для решения задачи один или несколько дочерних процессов. Взаимодействие между процессами осуществляется через системные сигналы/события и/или каналы (pipe). Необходимо обрабатывать системные ошибки, которые могут возникнуть в результате работы.

{\bfseries Вариант 6:} { Родительский процесс создает дочерний процесс. Первой строчкой пользователь в консоль родительского процесса вводит имя файла, которое будет использовано для открытия файла с таким именем на чтение. Стандартный поток ввода дочернего процесса перенаправляется в pipe1. Родительский процесс читает из pipe1 и прочитанное выводит в свой стандартный поток вывода. Родительский и дочерний процесс должны быть представлены разными программами. В файле записаны команды вида: "число число число <endline>". Дочерний процесс считает их сумму и выводит результат в стандартный поток вывода. Числа имеют тип int.}
{\bfseries Алгоритм решения задачи.} {В основном процессе до создания дочернего считаем имя и попробуем переопределить поток ввода. В случае успеха создадим канал для связи родительского и дочернего процесса. Т.к. в задании требуется представить дочерний процесс в отдельном файле, то используем execv для его запуска. В самом процессе будем считывать числа из файла, а писать результат вычислений в стандартный поток вывода (который на самом деле канал). Родительский процесс будет читать из канала и выводить на экран значение суммы.}
}
\pagebreak
