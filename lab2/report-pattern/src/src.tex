Листинг программы
\\Файл с основным процессом - main.c, файл с дочерним - child.c

\begin{lstlisting}[language=C]
//
// main.c
//

#include <stdio.h>
#include <unistd.h>

signed main() {
    char s [256];
    for (int i = 0; i < 256; i++) {
        s[i] = 0;
    }
    printf("Enter filename: ");
    scanf(" %s", s);
    FILE* input = NULL;
    input = freopen(s, "r", stdin);
    int fd[2];
    if (pipe(fd) == -1) {
        printf("Error creating pipe!");
        return 1;
    }
    int id = fork();
    if (id == -1) {
        printf("Error creating process!");
        return 2;
    }
    else if (id == 0) {
        close(fd[0]);
        if (input == NULL) {
            printf("Error opening file!\n");
            return 3;
        }
        if (dup2(fd[1], STDOUT_FILENO) == -1) {
            printf("Error changing stdout!\n");
            return 4;
        }
        char *argv[] = {"child", NULL};;
        if (execv("child", argv) == -1) {
            printf("Error executing child process!\n");
            return 5;
        }
    }
    else {
        close(fd[1]);
        int res;
        while (read(fd[0], &res, sizeof(int)) > 0) {
            printf("%d\n", res);
        }
        close(fd[0]);
    }
    return 0;
}
\end{lstlisting}

\begin{lstlisting}[language=C]
//
// child.c
//
#include <unistd.h>
#include <stdio.h>

int main(){
    char c;
    int num = 0, res = 0, minus = 0;
    while (scanf("%c", &c) > 0) {
        if (c == ' ' || c == '\t') {
            if (minus) {
                res = res - num;
            }
            else {
                res = res + num;
            }
            num = 0;
            minus = 0;
        }
        else if (c == '-') {
            minus = 1;
        }
        else if (c == '\n') {
            if (minus) {
                res = res - num;
            }
            else {
                res = res + num;
            }
            num = 0;
            minus = 0;
            write(STDOUT_FILENO, &res, sizeof(int));
            res = 0;
        }
        else if ('0' <= c && c <= '9') {
            num = num * 10 + c - '0';
        }
    }
    return 0;
}


\end{lstlisting}

\section{Тесты и протокол исполнения}
Заранее подготовим файл test.txt, из которого будет читать дочерний процессом
\\test.txt
\begin{alltt}
{10 5 25 10
0 6 -7
2 3 4}
\end{alltt}
\begin{alltt}
{\color{blue}(base) nikita@nikita-desktop:~/CLionProjects/OS_labs/lab2$ make}
cc -std=c99 main.c -o main
cc -std=c99 child.c -o child
{\color{blue}(base) nikita@nikita-desktop:~/CLionProjects/OS_labs/lab2$ ./main}
Enter filename: test.txt
50
-1
9
\end{alltt}
strace
\begin{lstlisting}
write(1, "Enter filename: ", 16Enter filename: )        = 16
read(0, test.txt
"test.txt\n", 1024)             = 9
lseek(0, -1, SEEK_CUR)                  = -1 ESPIPE (Invalid offset operation)
openat(AT_FDCWD, "test.txt", O_RDONLY)  = 3
dup3(3, 0, 0)                           = 0
close(3)                                = 0
pipe([3, 4])                            = 0
clone(child_stack=NULL, flags=CLONE_CHILD_CLEARTID|CLONE_CHILD_SETTID|SIGCHLD, child_tidptr=0x7fcdb8531810) = 6367
close(4)                                = 0
read(3, "2\0\0\0", 4)                   = 4
write(1, "50\n", 350
)                     = 3
--- SIGCHLD {si_signo=SIGCHLD, si_code=CLD_EXITED, si_pid=6367, si_uid=1000, si_status=0, si_utime=0, si_stime=0} ---
read(3, "\377\377\377\377", 4)          = 4
write(1, "-1\n", 3-1
)                     = 3
read(3, "\t\0\0\0", 4)                  = 4
write(1, "9\n", 29
)                      = 2
read(3, "", 4)                          = 0
close(3)                                = 0
exit_group(0)                           = ?
+++ exited with 0 +++

\end{lstlisting}
\pagebreak

